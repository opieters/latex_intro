\section{Introduction}

    Plants adapt their physiology continuously in response to fluctuations in the environment. This determines their performance, both in natural ecosystems, as well as in crop systems \citep{schurrFunctional2006,arsovaDynamics2020}. An increasing amount of experimental data suggests that proper acclimation of the photosynthesis biochemistry to environmental fluctuations is crucial for plant productivity, and potentially more important than high photosynthesis rates under steady-state conditions \citep{kaiserFluctuating2018,kromdijkImproving2016,vialet-chabrandImportance2017,matthewsAcclimation2018,townsendSuboptimal2018}.
    For instance, when focussing on the water relationships in a plant, changes in the environment cause variation in stomatal conductance which in turn determines leaf transpiration, and consequently leaf cooling as well as plant nutrient uptake.
    
    For example, because the response of the photosynthesis biochemistry to fluctuating light conditions is faster than the kinetics of stomatal conductance, these fluctuations also impact the interplay between plant water and carbon relations \citep{lawsonImproving2012,lawsonStomatal2014}. Consequently, a mismatch arises between CO\textsubscript{2} assimilation and water loss \citep{mcauslandEffects2016}. Reducing this mismatch, and improving the capacity of crop photosynthesis to respond to fluctuating light environments is, therefore, a promising avenue for breeding more productive crop varieties \citep{salterRate2019,murchieDynamic2020}. 
    
    Given the importance of plant physiological responses to environmental fluctuations,  it is essential that new field phenotyping technologies specifically focus on capturing such fast-changing dynamics \citep{murchieMeasuring2018}. Yet, it remains difficult to capture plant photosynthetic and water status responses to fluctuating conditions in the field. Gas exchange devices based on infrared gas analysers (IRGA) allow continuous measurements of transpiration and CO\textsubscript{2} assimilation and capture detailed dynamics \citep{kromdijkImproving2016}. However, this approach does not allow for high-throughput measurements and requires expensive devices. Furthermore, these systems monitor individual leaves and do not provide concurrent data at the plant scale, while recent evidence points out that plants display systemic responses under fluctuating light conditions \citep{shimadzuWhole2019}. 
    
    Chlorophyll fluorescence imaging is a powerful method to monitor the photosynthetic capacity of plants \citep{bakerChlorophyll2008,murchieChlorophyll2013}. However, chlorophyll fluorescence measurements typically require a dark adaptation period of one hour, which limits the applicability to study short-term dynamics. New developments in chlorophyll fluorescence imaging methods like Light-Induced Fluorescence Transient (LIFT) or Sun Induced Fluorescence (SIF) overcome this dark adaptation period and can be used as proxies. These methods can be applied at different scales and show great promise, though they do not enable acquisition of absolute photosynthesis biochemistry data, and still require extensive calibration \citep{murchieDynamic2020,bandopadhyayReview2020}.
    
    Moreover, chlorophyll fluorescence imaging is unable to monitor stomatal conductance. Because stomatal conductance is closely related to leaf temperature, thermal sensors can be used to monitor it by applying basic energy balance equations \citep{jonesIrrigation2004,maesEstimating2012}. These equations require the assessment of the micro-environmental conditions of the leaf and the boundary layer resistance to water vapour \citep{jonesUse2002}. Although most studies with thermal sensors use single time point observations, continuous monitoring of dynamic stomatal conductance in response to a fluctuating environment is possible and can be combined with chlorophyll fluorescence imaging to link plant water relations and photosynthesis \citep{mcauslandEffects2016}. 
    
    Generally, field phenotyping uses imagery that captures the plants' reflectance in different wavelengths. This information can be used to determine specific plant traits. Examples include, but are not limited to, detection of biotic and abiotic stress, and estimation of nitrogen content and yield. \citet{mirHighthroughput2019} provides an overview of current methods. In this respect, broadband RGB cameras are often used in phenotyping experiments because they are inexpensive and can be used to monitor plant growth at the scale of days and weeks, or to develop spectral indices referring to the greenness or canopy cover \citep{borra-serranoClosing2020}. However, these sensors do not provide information on dynamic responses of photosynthesis over time scales of seconds or minutes.
    
    Hyperspectral imaging sensors capture reflectance in many wavelengths and are increasingly applied in phenotyping research. Hyperspectral imaging has already been applied to various settings that benefit from higher spectral resolutions to detect biotic and abiotic influences on plants \citep{khanModern2018}. Examples of studies on biotic factors include blight caused by \textit{Alternaria solani} in potato \citep{vandevijverInfield2020}, late blight caused by \textit{Phytophthora infestans} in potato \citep{franceschiniFeasibility2019}, or tracking the development of three foliar diseases in barley \citep{wahabzadaPlant2016}. \citet{mahleinPlant2015} and \citet{loweHyperspectral2017} provide comprehensive overviews of plant disease detection using imaging sensors and hyperspectral sensors specifically. Studies in which hyperspectral imaging was used to investigate plant responses in interaction with abiotic factors include, for example, detection of green citrus fruits on trees \citep{okamotoGreen2009}, nitrogen deficiency in sorghum \citep{zhaoNitrogen2005}, seasonal structural changes and a heterogeneous architecture in an olive orchard \citep{zarco-tejadaSpatiotemporal2013}, nitrogen and water distribution quantification in wheat \citep{bruningDevelopment2019}, and drought stress in barley and saxaul \citep{behmannDetection2014,jinHyperspectral2016}. One common aspect that all the aforementioned studies share, is the presence of a clear treatment or perturbation that results in substantial stress or that influences the phenology of the plant. Furthermore, they usually monitor plants over extended periods, often months or an entire growing season, typically with several days to a week between measurements. 
    
    However, to the best of our knowledge, there is no research yet on hyperspectral data concerning photosynthetic activity at high spatial and temporal resolution in the seconds to minutes range. Nonetheless, vegetation indices (VI) and data derived from hyperspectral cameras might have the potential to monitor subtle dynamics on a detailed scale. For instance, the Photochemical Reflectance Index (PRI) offers potential if it can capture variations in the de-epoxidation of the xanthophyll cycle \citep{alonsoDiurnal2017} or the Canopy Chlorophyll Content Index (CCCI) which offers a good measure of canopy nitrogen \citep{barnesCoincident2000}. 
    
    We believe that new phenotyping technologies should increasingly focus on capturing dynamics in photosynthesis biochemistry and stomatal conductance kinetics under stress-free yet fluctuating conditions. The objective of the present study is to evaluate the potential of hyperspectral snapshot cameras for this purpose. More specifically, we aim to capture the dynamic responses of strawberry plants in fluctuating, yet stress-free environmental conditions. To this end, experiments were conducted in growth chambers because they offer excellent controllability of the environment over field experiments.