% !TeX document-id = {2b322129-792a-4592-a97a-4ee984dc1dbc}
% !TeX spellcheck = en_GB
% !BIB program = bibtex
% !TeX TS-program = pdflatex

\documentclass[10pt,authoryear,a4paper]{elsarticle}

\usepackage{geometry}
\newgeometry{margin=2.25cm}

\usepackage{hyperref}

\usepackage{amsmath}
\usepackage{xfrac}
\providecommand\ddfrac[2]{\frac{\displaystyle #1}{\displaystyle #2}}

\usepackage{cleveref}

\usepackage{nag}


\usepackage[binary-units]{siunitx}
\sisetup{group-separator = {,}}

\usepackage{pgfplotstable}
\pgfplotsset{compat=1.16} 


\usepackage{listofitems}
\usepackage{booktabs}

\newcolumntype{L}[1]{>{\raggedright\let\newline\\\arraybackslash\hspace{0pt}}m{#1}}
\newcolumntype{C}[1]{>{\centering\let\newline\\\arraybackslash\hspace{0pt}}m{#1}}
\newcolumntype{R}[1]{>{\raggedleft\let\newline\\\arraybackslash\hspace{0pt}}m{#1}}

\usepackage{makecell}
\renewcommand\theadfont{\bfseries}

\usepackage{setspace}

\usepackage{tabularx}
\usepackage{placeins}


\usepackage{cancel}

\usepackage{colortbl}

\newcommand{\VPDL}{$VPD_\text{leaf}$}
\newcommand{\Tair}{$T_\text{air}$}
\newcommand{\Tleaf}{$T_\text{leaf}$}
\newcommand{\Cond}{$g_\text{s}$}
\newcommand{\Photo}{$P_\text{n}$}
\newcommand{\Transp}{$Tr$}
\newcommand{\RH}{$RH$}
\newcommand{\PAR}{$PAR$}

\journal{Computers and Electronics in Agriculture}

\usepackage{placeins}

\begin{document}

\begin{frontmatter}

\title{Limitations of snapshot hyperspectral cameras to monitor plant response dynamics in stress-free conditions}


\author[a,b]{Olivier~Pieters\corref{cor1}}
\ead{olivier.pieters@ugent.be}
\ead[url]{olivierpieters.be}

\author[b]{Tom~De~Swaef}
\ead{tom.deswaef@ilvo.vlaanderen.be}

\cortext[cor1]{Corresponding author}

\address[a]{IDLab-AIRO -- Ghent University -- imec, Technologiepark-Zwijnaarde 126, 9052 Zwijnaarde, Belgium}
\address[b]{Plant Sciences Unit, Flanders Research Institute for Agriculture, Fisheries and Food, Caritasstraat 39, 9090 Melle, Belgium}

\begin{abstract}  
    In perennial ryegrass breeding programs, dry matter yield (DMY) of
    individual plots is monitored destructively at the different cuts or
    derived from non-destructive canopy height measurements using devices
    like rising plate meters (RPM). These approaches both have constraints.
    Destructive sampling implies low temporal resolution, restraining the
    study of dry matter accumulation rates, while RPM measurements are
    influenced by the canopy structure and limit intra-field variability
    identification. We present a phenotyping methodology, based on the use
    of an affordable RGB camera mounted on an Unmanned Aerial Vehicle (UAV),
    to monitor the spatial and temporal evolution of canopy height and to
    estimate DMY. Weekly flights were carried out from April to October
    above a field comprising a diverse set of accessions. To test the
    capability of the model extracted from the data to estimate canopy height, 
    8 ground control points and 28 artificial height references were placed 
    at different locations. Accurate flights with an RMSE as low as 0.94 cm were
    achieved. In addition, canopy height was recorded using an RPM and
    destructive biomass samples were collected. Different models (linear,
    multiple linear, principal components, partial least squares regression
    and random forest) were used to predict DMY and their performance was
    evaluated. The best estimations were obtained by combining variables
    including canopy height, vegetation indices and environmental data in a
    multiple linear regression (${R^2 = 0.81}$). All models built using UAV data
    obtained a lower RMSE than the one using RPM data. The approach
    presented is a possibility for breeders to incorporate new information
    in their selection process.
\end{abstract}

\begin{highlights}
    \item Study of eco-physiological plant response dynamics in stress-free conditions
    \item Leaf temperature and vapour pressure deficit were estimated with $\text{NMSE} < 0.23$
\end{highlights}

\begin{keyword}
Hyperspectral camera\sep phenotyping\sep proximal sensing\sep dynamic\sep monitoring 
\end{keyword}

\end{frontmatter}

\end{document}